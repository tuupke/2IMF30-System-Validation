% !TEX root = ../report.tex
\cbstart
Our model consists of 6 parallel processes that communicate.
\begin{enumerate}
 \item A process \emph{WaferTracker}, which tracks the positions and state of each wafer. This process communicates with the three processes controlling the robots and projector.
 \item A process \emph{RobotOutside}, controlling \emph{R1}.
 \item A process \emph{RobotInside}, controlling \emph{R2}.
 \item A process \emph{ProjectorRobot}, controlling \emph{R3} and the projector.
 \item Two identical processes Sluice managing a single sluice each. They steer the sluice doors, the pumps and communicate with the robots about whether or not they may reach into the sluice.
\end{enumerate}

\subsection{Process \emph{WaferTracker}}

\emph{WaferTracker} keeps track of the place statuses. There are two kind of status
updates: single place status updates and move statues updates. When a
process wants to update the status of a single place it sends a
\emph{StatusUpdateRequest} with parameters place and the 'from status'. When there
is an unlocked place with that state, the \emph{WaferTracker} can receive the
\emph{StatusUpdateRequest} and consequently locks that state. The \emph{WaferTracker} is
then able to send back the \emph{StatusUpdateResponse} with all possible 'to
statuses', which the other process will receive after it has done possible
other necessary actions. The \emph{WaferTracker} will then unlock the corresponding
state again. A move status update works similar, however it works with a
\emph{MoveRequest} that has two place parameters and one status parameter. The
conditions are that the 'to' place must be empty and the 'from' place must
have the given status. Of course both states must be unlocked. When two
places a and b are locked, the \emph{WaferTracker} can send a \emph{MoveResponse} with
the two places, which the other process will receive when the actual move
action is done.

\emph{MoveRequest}s to a sluice have extra conditions: other sluices must not
contain a wafer with the same status and additionally, if the status is
unprojected, there must be at least one place inside that is empty.

\subsection{Process \emph{RobotOutside}}

 \emph{RobotOutside} performs \emph{detectInputWafer} actions and moves unprojected wafers
 from \emph{RI} to a sluice and projected wafers from sluices to \emph{RO}. It can also
 move unprojected wafers from a broken sluice to another sluice. When there
 is a projected wafer in \emph{RO}, it can remove that wafer from the system. The
 variables limited, limit and repeat can be used to limit the number of
 wafers (useful while debugging). When limited is false, wafers can always
 arrive (provided that \emph{RI} is empty). Otherwise, when repeat is false, the
 maximum number of \emph{detectInputWafer} actions that can be executed is equal to
 the variable limit; when repeat is true, the maximum number of wafers that
 is present in the system simultaneously is equal to the variable limit.
 In production, the limited variable must be set to false.
 
 \emph{RobotOutside} makes use of three terminating auxiliary processes: \emph{MoveToSluice}, \emph{MoveFromSluice}, and \emph{MoveFromSluiceToSluice}
 
\subsection{Process \emph{RobotInside}}

 \emph{RobotInside} can move projected wafers from a sluice to \emph{O} and unprojected
 wafers from \emph{I} to a sluice. It can also move projected wafers from a broken
 sluice to another sluice.

 \subsection{Auxiliary Processes \emph{MoveToSluice}, \emph{MoveFromSluice}, and \emph{MoveFromSluiceToSluice}}

 \emph{RobotOutside} makes use of three terminating auxiliary processes: \emph{MoveToSluice}, \emph{MoveFromSluice}, and \emph{MoveFromSluiceToSluice}.
 
  \emph{MoveToSluice} is a terminating auxiliary process used by \emph{RobotInside} and
 \emph{RobotOutside}. It is able to execute a multi-action, consisting of sending
 a \emph{SluiceRequest} and a \emph{MoveRequest}. It can also receive a \emph{SluiceResponse}.
 This means that the \emph{SluiceRequest} is processed and the specified door is
 open. Then it can perform a move action to that sluice and consequently
 send a \emph{SluiceMoveDone} message and receive a \emph{MoveResponse} simultaneously, to
 indicate that the sluice is allowed to close its door and the statuses can
 be updated by the \emph{WaferTracker}. It can also receive a \emph{SluiceFailedResponse}.
 This means that something wrong happened with the sluice door and a move
 action is not possible. It then sends a \emph{MoveCancel} message to the
 \emph{WaferTracker} to unlock the places again.



 \emph{MoveFromSluice} is imilar to \emph{MoveToSluice}, but the move is in the other direction.


 \emph{MoveFromSluiceToSluice} is similar to \emph{MoveToSluice} and \emph{MoveFromSluice}, but now the move is from a
 sluice to a sluice. Moreover, the first multi-action includes now a
 \emph{SluiceStuck} action too, which indicates that at least one door is broken and
 it specifies which doors are currently open. The remaining actions are
 similar to \emph{MoveToSluice} and \emph{MoveFromSluice}.

 
\subsection{Process \emph{ProjectorRobot}}


 \emph{ProjectorRobot} can move unprojected wafers from \emph{I} to \emph{P} and projected wafers
 from \emph{P} to \emph{O}. It can also execute \emph{beginProject} when an unprojected wafer is
 at \emph{P}, changing the status of \emph{P} into projecting and it can execute \emph{endProject}
 when the status of \emph{P} is projecting.

\subsection{Process Sluice}


 \emph{Sluice} keeps track of the state of one sluice: which doors are open and
 which doors are stuck. It can only receive a \emph{SluiceRequest} when both doors
 are not broken. It will make sure that the specified door will be opened,
 while taking care of closing other doors and making the air pressure right.
 Then, it sends a \emph{SluiceResponse}, indicating that the move action can be
 executed, and afterwards receives a \emph{SluiceMoveDone} message. It any time that
 a door can be opened or closed, it is also possible to send a doorStuck
 action. This results in a broken door and a \emph{SluiceFailedResponse} will be
 sent, instead of a \emph{SluiceResponse}, indicating that the \emph{MoveRequest} has to
 be canceled.
 
 In the case that one door is broken and the other is not, it tries to open
 the door that is not broken, in order to give the robots the opportunity
 to move a wafer from a broken sluice to the other.
 It is also possible to receive a \emph{SluiceStuck} message when one of the doors
 is stuck. This makes it possible for other processes to know if a sluice is
 broken.
 
 When both doors are closed and stuck, it will mark that place as stuck. This
 necessary because there is a condition in the \emph{WaferTracker} that makes it
 impossible to move a wafer to a sluice when there is another sluice
 containing a wafer with the same status.

\cbend