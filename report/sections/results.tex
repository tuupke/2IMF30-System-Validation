% !TEX root = ../report.tex
\cbstart


\subsection{Method}
All results were obtained using version 201409.1.13218 of mCRL2, the commands used are listed below.

At first the model has to be converted from the mCRL2 format to \emph{LPS} using the command:
\begin{lstlisting}[language=bash]
mcrl22lps model.mcrl2 model.lps --lin-method=regular --rewriter=jitty --verbose
\end{lstlisting}

Afterwards each requirement can be checked using the following two commands. In these commands \emph{$<$REQ$>$} should be replaced by the requirement number which has to get tested.

\begin{lstlisting}[language=bash]
lps2pbes model.lps model.req<REQ>.pbes --formula=Requirements/Requirement<REQ>.mcf --verbose
pbes2bool model.req<REQ>.pbes --rewriter=jittyc --strategy=0 --verbose
\end{lstlisting}

\subsection{Output}

The commands above produced the following outputs.


\begin{tabular}{|l|l|}
\hline
Requirement 1 & true \\
\hline
Requirement 2 & true \\
\hline
Requirement 3 & true \\
\hline
Requirement 4 & true \\
\hline
Requirement 5 & true \\
\hline
Requirement 6 & true \\
\hline
Requirement 7 & true \\
\hline
Requirement 8 & true \\
\hline
Requirement 9 & true \\
\hline
Requirement 10 & true \\
\hline
\end{tabular}


All of them return true, therefore we can guarantee that the model meet all our requirements.

\cbend