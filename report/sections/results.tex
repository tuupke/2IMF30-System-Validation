% !TEX root = ../report.tex
\cbstart

\begin{tabular}{|l|l|}
\hline
Requirement 1 & true \\
\hline
Requirement 2 & true \\
\hline
Requirement 3 & true \\
\hline
Requirement 4 & true \\
\hline
Requirement 5 & not yet known \\
\hline
Requirement 6 & true \\
\hline
Requirement 7 & true \\
\hline
Requirement 8 & true \\
\hline
Requirement 9 & true \\
\hline
Requirement 10 & true \\
\hline
\end{tabular}

\subsection{Method}
All results were obtained using version \todo{versie nummer hier} of MCRL2, the commands used are listed below.

At first the model has to be converted from the MCRL2 format to \emph{LPS} using the command:
\begin{lstlisting}[language=bash]
mcrl22lps model.mcrl2 model.lps --lin-method=regular --rewriter=jitty --verbose
\end{lstlisting}

Afterwards each requirement can be checked using the following two commands. In these commands \emph{$<$REQ$>$} should be replaced by the requirement number which has to get tested.

\begin{lstlisting}[language=bash]
lps2pbes model.lps model.req<REQ>.pbes --formula=Requirements/Requirement<REQ>.mcf --verbose
pbes2bool model.req<REQ>.pbes --rewriter=jittyc --strategy=0 --verbose
\end{lstlisting}

\cbend