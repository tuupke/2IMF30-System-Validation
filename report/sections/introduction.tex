The goal of the assignment is to apply the techniques and tools to the design of the software controller of a small distributed and/or embedded system. The purpose is to design this system such that it is proven to comply with all the requirements which must have been formulated in advance.\\

At first, we will give a short textual description of the system that we chose to design and implement. After that we will give a list of requirements for the whole system, that we derived from the description. These requirements are described in natural language. When we have settled all the requirements, we continue with identifying the interactions that are relevant to the system. We describe clearly but compactly the meaning of each interaction in words, together with its desired parameters. When the interactions have been described, we can start with translating all the global requirements in terms of these interactions. When the requirements have been described in this way, we can start using the modal $\mu$-calculus to convert the requirements into modal formulas with which we can verify our system. After this, we will describe the behaviour of all controllers in the architecture using mCRL2. This gives a short overview of all the parallel processes that communicate in our system. Finally, we will argue which requirements we have verified and which requirements we were unable to verify.