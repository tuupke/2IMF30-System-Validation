% !TEX root = ../report.tex

\newcommand{\tab}{\hspace{1em}}

This section contains a translation of the requirements from section 5 to a language that enables their unambiguous verification. Each requirement is expressed as a formula in modal $\mu$-calculus, these type of formulas can be mathematically tested against the controller model defined in section 7 to produce the results of section 8. 

\begin{description}
 \item[1. At least one door of a sluice is closed at any point in time]\mbox{}\\
\[
[true^*]\forall s:Sluice,d:Door.[openDoor(d, s) \cdot \overline{closeDoor(d, s)}^* \cdot openDoor(d, s)]\mathit{false}
\]


 \item[2. No robot may interact with a sluice whenever its access door is closed]\mbox{}\\
\begin{align*}
& [true^*]\forall s:Sluice.( \\
& \tab [closeDoor(outsideDoor, s).(\overline{openDoor(outsideDoor, s)}^*.( \\
& \tab\tab move(RI, sluicePlace(s)) \cup move(sluicePlace(s), RI) \cup \\
& \tab\tab move(RO, sluicePlace(s)) \cup move(sluicePlace(s), RO) \\
& \tab )]false \wedge \\
& \tab [closeDoor(insideDoor, s).(\overline{openDoor(insideDoor, s)}^*.( \\
& \tab\tab move(I, sluicePlace(s)) \cup move(sluicePlace(s), I) \cup \\
& \tab\tab move(O, sluicePlace(s)) \cup move(sluicePlace(s), O) \\
& \tab )]\mathit{false} \\
& )
\end{align*}
 
 \item[\smash{\shortstack[l]{3. Any wafer in the system will eventually be processed unless for each sluice at \\ least one door gets stuck, or the wafer is in a sluice with two broken, closed doors}}]
 
This requirement is quite complex to express in modal $\mu$-calculus. In the first place, we need to keep track for each door whether it is open and whether it is stuck. We do this with a greatest fixed point $X$ with two functions $openDoors$ and $brokenDoors$. We update those functions when $closeDoor$, $openDoor$, or $doorStuck$ actions happen. Whenever a $detectInputWafer$ occurs, we need to keep track of the wafer that enters the system. We do this with a least fixed point $Y$ (note that this is nested within $X$). However, within this fixed point we need to continue tracking the doors. So we use $openDoorsLocal$ and $brokenDoorsLocal$ as data within this fixed point $Y$, take over the values of $openDoors$ and $brokenDoors$ and update the local versions of those functions when necessary in this least fixed point. We also need to keep track of position of the wafer, and whether is $projected$ or currently $projecting$. We use $position$, $projected$, and $projecting$ for this. We update this data when $move$, $beginProject$ and $endProject$ actions occur. Finally, the least fixed point can get a true value when $position$ equals $RO$, or when each sluice has at least one broken door, or when the position is a sluice of which all doors are closed and broken.
 
\begin{align*}\allowdisplaybreaks
& \nu X(openDoors: Sluice \to (Door \to Bool) = sluiceDoorsFalse, \\
& \tab\tab brokenDoors: Sluice \to (Door \to Bool) = sluiceDoorsFalse) \\
& \tab . ([\overline{\exists d: Door, s: Sluice . openDoor(d, s)} \\
& \tab\tab\tab \cap \overline{\exists d: Door, s: Sluice . closeDoor(d, s)} \\
& \tab\tab\tab \cap \overline{exists d: Door, s: Sluice . doorStuck(d, s)} \\
& \tab\tab\tab \cap \overline{detectInputWafer}]X(openDoors, brokenDoors) \\
& \tab\tab \wedge \forall d: Door, s: Sluice . ( \\
& \tab\tab\tab [openDoor(d, s)]X(openDoors[s \to openDoors(s)[d \to true]], brokenDoors) \\
& \tab\tab\tab \wedge [closeDoor(d, s)]X(openDoors[s \to openDoors(s)[d \to \mathit{false}]], brokenDoors) \\
& \tab\tab\tab \wedge [doorStuck(d, s)]X(openDoors, brokenDoors[s \to brokenDoors(s)[d \to \mathit{false}]])) \\
& \tab\tab \wedge [detectInputWafer]\mu Y( \\ 
& \tab\tab\tab\tab\tab position: Place = RI, \\
& \tab\tab\tab\tab\tab projecting: Bool = \mathit{false}, \\
& \tab\tab\tab\tab\tab projected: Bool = \mathit{false}, \\
& \tab\tab\tab\tab\tab openDoorsLocal: Sluice \to (Door \to Bool) = openDoors, \\
& \tab\tab\tab\tab\tab brokenDoorsLocal: Sluice \to (Door \to Bool) = brokenDoors \\
& \tab\tab\tab\tab ) . (([\overline{\exists a, b: Place . move(a, b)} \\
& \tab\tab\tab\tab\tab\tab \cap \overline{beginProject} \\
& \tab\tab\tab\tab\tab\tab \cap \overline{endProject} \\
& \tab\tab\tab\tab\tab\tab \cap \overline{\exists d: Door, s: Sluice . doorStuck(d, s)} \\
& \tab\tab\tab\tab\tab\tab \cap \overline{\exists d: Door, s: Sluice . openDoor(d, s)} \\
& \tab\tab\tab\tab\tab\tab \cap \overline{\exists d: Door, s: Sluice . closeDoor(d, s)}]Y( \\
& \tab\tab\tab\tab\tab\tab\tab\tab\tab\tab position, projecting, projected, openDoorsLocal, brokenDoorsLocal \\
& \tab\tab\tab\tab\tab\tab\tab\tab ) \\
& \tab\tab\tab\tab\tab\tab  \wedge \forall a, b: Place . [move(a, b)]Y( \\
& \tab\tab\tab\tab\tab\tab\tab\tab\tab\tab if(a \approx position, b, position), \\
& \tab\tab\tab\tab\tab\tab\tab\tab\tab\tab projecting \wedge a \not\approx position, \\
& \tab\tab\tab\tab\tab\tab\tab\tab\tab\tab projected, \\
& \tab\tab\tab\tab\tab\tab\tab\tab\tab\tab openDoorsLocal, \\
& \tab\tab\tab\tab\tab\tab\tab\tab\tab\tab brokenDoorsLocal \\
& \tab\tab\tab\tab\tab\tab\tab\tab ) \\
& \tab\tab\tab\tab\tab\tab  \wedge [beginProject]Y( \\
& \tab\tab\tab\tab\tab\tab\tab\tab\tab\tab position, \\
& \tab\tab\tab\tab\tab\tab\tab\tab\tab\tab position \approx P, \\
& \tab\tab\tab\tab\tab\tab\tab\tab\tab\tab projected, \\
& \tab\tab\tab\tab\tab\tab\tab\tab\tab\tab openDoorsLocal, \\
& \tab\tab\tab\tab\tab\tab\tab\tab\tab\tab brokenDoorsLocal \\
& \tab\tab\tab\tab\tab\tab\tab\tab ) \\
& \tab\tab\tab\tab\tab\tab  \wedge [endProject]Y( \\
& \tab\tab\tab\tab\tab\tab\tab\tab\tab\tab position, \\
& \tab\tab\tab\tab\tab\tab\tab\tab\tab\tab \mathit{false}, \\
& \tab\tab\tab\tab\tab\tab\tab\tab\tab\tab projected \vee (position \approx P \wedge projecting), \\
& \tab\tab\tab\tab\tab\tab\tab\tab\tab\tab openDoorsLocal, \\
& \tab\tab\tab\tab\tab\tab\tab\tab\tab\tab brokenDoorsLocal \\
& \tab\tab\tab\tab\tab\tab\tab\tab ) \\
& \tab\tab\tab\tab\tab\tab  \wedge \forall d: Door, s: Sluice . ( \\ 
& \tab\tab\tab\tab\tab\tab\tab [doorStuck(d, s)]Y( \\
& \tab\tab\tab\tab\tab\tab\tab\tab\tab\tab\tab position, \\
& \tab\tab\tab\tab\tab\tab\tab\tab\tab\tab\tab projecting, \\
& \tab\tab\tab\tab\tab\tab\tab\tab\tab\tab\tab projected, \\
& \tab\tab\tab\tab\tab\tab\tab\tab\tab\tab\tab openDoorsLocal, \\
& \tab\tab\tab\tab\tab\tab\tab\tab\tab\tab\tab brokenDoorsLocal[s \to brokenDoorsLocal(s)[d \to true]] \\
& \tab\tab\tab\tab\tab\tab\tab\tab\tab ) \\
& \tab\tab\tab\tab\tab\tab\tab  \wedge [closeDoor(d, s)]Y( \\
& \tab\tab\tab\tab\tab\tab\tab\tab\tab\tab\tab position, \\
& \tab\tab\tab\tab\tab\tab\tab\tab\tab\tab\tab projecting, \\
& \tab\tab\tab\tab\tab\tab\tab\tab\tab\tab\tab projected, \\
& \tab\tab\tab\tab\tab\tab\tab\tab\tab\tab\tab openDoorsLocal[s \to openDoorsLocal(s)[d \to true]], \\
& \tab\tab\tab\tab\tab\tab\tab\tab\tab\tab\tab brokenDoorsLocal \\
& \tab\tab\tab\tab\tab\tab\tab\tab\tab ) \\
& \tab\tab\tab\tab\tab\tab\tab \wedge [openDoor(d, s)]Y( \\
& \tab\tab\tab\tab\tab\tab\tab\tab\tab\tab\tab position, \\
& \tab\tab\tab\tab\tab\tab\tab\tab\tab\tab\tab projecting, \\
& \tab\tab\tab\tab\tab\tab\tab\tab\tab\tab\tab projected, \\
& \tab\tab\tab\tab\tab\tab\tab\tab\tab\tab\tab openDoorsLocal[s \to openDoorsLocal(s)[d \to \mathit{false}]], \\
& \tab\tab\tab\tab\tab\tab\tab\tab\tab\tab\tab brokenDoorsLocal \\
& \tab\tab\tab\tab\tab\tab\tab\tab\tab ) \\
& \tab\tab\tab\tab\tab\tab ) \\
& \tab\tab\tab\tab\tab\tab \wedge \langle true \rangle true \\
& \tab\tab\tab\tab\tab ) \\
& \tab\tab\tab\tab\tab \vee (position \approx RO \wedge projected) \\
& \tab\tab\tab\tab\tab \vee (\forall s: Sluice . \exists d: Door . brokenDoorsLocal(s)(d)) \\
& \tab\tab\tab\tab\tab \vee (\exists s: Sluice . position \approx sluicePlace(s) \\
& \tab\tab\tab\tab\tab\tab\tab \wedge \forall d: Door . !openDoorsLocal(s)(d) \wedge brokenDoorsLocal(s)(d)) \\
& \tab\tab ) \\
& \tab ) \\
\end{align*}

 
 \item[4. Internal racks, sluices and the projector each contain at most one wafer]
\begin{align*}
& \nu X(occupied:Place \to Bool=placesFalse).(\\
& \tab\tab [\overline{exists a,b:Place.move(a,b)} \wedge \overline{detectInputWafer}]X(occupied) \\
& \tab\tab \wedge [detectInputWafer]X(occupied[RI \to true]) \\
& \tab\tab \wedge \forall a,b:Place.[move(a,b)](val(b==RO || (occupied(a) \implies \overline{occupied(b)})) \wedge \\
& \tab\tab X(occupied[b \to occupied(a)][a \to \mathit{false}])) \\
& )
\end{align*}

 
 \item[5. When the projector is at work, no interaction with the wafer is permissible]
 	\begin{align*}
 		&[true^*.beginProject().\overline{endProject()}^*.move(P,O)]\mathit{false}
	\end{align*}
	
 \item[6. A sluice door cannot open until the pressure on both sides is equal]
	\begin{align*}
& \forall s: Sluice . \nu X(pressure: Pressure = normal) . ( \\
&   [\overline{\exists d: Door . openDoor(d, s)} \cap \overline{\exists p: Pressure . readAirPressure(s, p)}]X(pressure) \\
&   \wedge [openDoor(insideDoor, s)](pressure \approx vacuum \wedge X(pressure)) \\
&   \wedge [openDoor(outsideDoor, s)](pressure \approx normal \wedge X(pressure)) \\
&   \wedge \forall p: Pressure . [readAirPressure(s, p)]X(p) \\
& ) \\
	\end{align*}
	
 \item[7. Sluice pumps cannot operate until both of its doors are closed]
 
 \begin{align*}
 &[true^*]\forall s:Sluice, d:Door . ( \\
 &\tab [openDoor(d,s).\overline{closeDoor(d,s)}*.vacuum(s)]\mathit{false} \\
 &\tab \wedge [openDoor(d,s).\overline{closeDoor(d,s)}*.deVacuum(s)]\mathit{false} \\
 &)
 \end{align*}

 \item[8. No robot can place a projected wafer in $RI$]

\begin{align*}
&[true^*] \nu X(status:Place \to WaferState=emptyPlaces).( \\
&\tab  [\overline{(\exists a,b:Place.move(a,b)) \vee detectInputWafer \vee endProject}]X(status) \\
&\tab  \wedge [detectInputWafer]X(status[RI \to unprojected]) \\
&\tab  \wedge \forall a,b:Place.[move(a, b)](((b \approx RI) \implies (status(a) \not \approx projected)) \\
&\tab\tab\tab \wedge X(status[b \to status(a)][a \to empty])) \\
&\tab  \wedge [endProject]X(status[P \to projected]) \\
&)
\end{align*}

 \item[9. No robot can place an unprojected wafer in $RO$]

\begin{align*}
&[true^*] \nu X(status:Place \to WaferState=emptyPlaces).( \\
&\tab  [\overline{(\exists a,b:Place.move(a,b)) \vee detectInputWafer \vee endProject}]X(status) \\
&\tab  \wedge [detectInputWafer]X(status[RI \to unprojected]) \\
&\tab  \wedge \forall a,b:Place.[move(a, b)](((b \approx RO) \implies (status(a) \not \approx unprojected)) \\
&\tab\tab\tab \wedge X(status[b \to status(a)][a \to empty])) \\
&\tab  \wedge [endProject]X(status[P \to projected]) \\
&)
\end{align*}

 \item[10. No robot can take a wafer from $RO$]

\[
	\forall p:Place . [true^* \cdot move(RO, p)]\mathit{false}
\]

\end{description}
