% !TEX root = ../report.tex
\subsection{Definitions}
\begin{enumerate}
  \item A wafer is considered projected when the projector has finished projecting the wafer.
  \item A wafer is considered processed when the wafer is projected and the wafer has been placed in \textit{RO}.
  \item A sluice is considered broken if one or more of its doors are stuck.
  \item The system is considered broken when both sluices are broken.
\end{enumerate}

\subsection{Assumptions}
\begin{enumerate}
  \item The doors are the only part of the machines that can malfunction. They can get stuck at any point in time. Since opening and closing a door is assumed to be an atomic action taking no time, doors only get stuck in either an open position or a closed position, never a half open position.
  \item The main room is assumed to always be a vacuum.
  \item The external input and output racks are assumed to be infinite in size.
  \item Sluice pumps change the air pressure to be normal or vacuum.
  \item The physical waferstepper is initialised to a state where all doors are closed and no wafers are in the system.
\end{enumerate}

\subsection{Requirements}
\begin{enumerate}
  \item At least one door of a sluice is closed at any point in time.
  \item No robot may interact with a sluice whenever its access door is closed.
  \item Any wafer in the system will eventually be processed unless the system breaks or the wafer is inside a sluice when the sluice itself breaks.
  \item Internal racks, sluices, and the projector place each contain at most one wafer.
  \item When the projector is at work, no interaction with the wafer is permissible.
  \item A sluice door cannot open until the pressure on both sides is equal.
  \item Sluice pumps will not operate until both of its doors are closed.
  \item No robot can place a projected wafer in \textit{RI}.
  \item No robot can place an unprojected wafer in \textit{RO}.
  \item No robot can take a wafer from \textit{RO}.
\end{enumerate}