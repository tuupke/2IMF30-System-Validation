	\documentclass[a4paper,twoside,11pt]{article}
\usepackage{a4wide,graphicx,fancyhdr,amsmath,amssymb,float,longtable,chronology,caption,subcaption,appendix}
\usepackage{algorithmic}
\usepackage{hyperref}
\usepackage{listings}
\usepackage{url}
\usepackage{pgffor}
\usepackage{color}

%----------------------- Macros and Definitions --------------------------

\definecolor{mygreen}{rgb}{0,0.6,0}
\definecolor{mygray}{rgb}{0.5,0.5,0.5}
\definecolor{mymauve}{rgb}{0.58,0,0.82}

\lstset{ %
  backgroundcolor=\color{white},   % choose the background color; you must add \usepackage{color} or \usepackage{xcolor}
  basicstyle=\footnotesize,        % the size of the fonts that are used for the code
  breakatwhitespace=false,         % sets if automatic breaks should only happen at whitespace
  breaklines=true,                 % sets automatic line breaking
  captionpos=b,                    % sets the caption-position to bottom
  commentstyle=\color{mygreen},    % comment style
  deletekeywords={...},            % if you want to delete keywords from the given language
  escapeinside={\%*}{*)},          % if you want to add LaTeX within your code
  extendedchars=true,              % lets you use non-ASCII characters; for 8-bits encodings only, does not work with UTF-8
  frame=single,	                   % adds a frame around the code
  keepspaces=true,                 % keeps spaces in text, useful for keeping indentation of code (possibly needs columns=flexible)
  keywordstyle=\color{blue},       % keyword style
 % language=mcrl2,                 % the language of the code
  otherkeywords={*,...},           % if you want to add more keywords to the set
  numbers=left,                    % where to put the line-numbers; possible values are (none, left, right)
  numbersep=5pt,                   % how far the line-numbers are from the code
  numberstyle=\tiny\color{mygray}, % the style that is used for the line-numbers
  rulecolor=\color{black},         % if not set, the frame-color may be changed on line-breaks within not-black text (e.g. comments (green here))
  showspaces=false,                % show spaces everywhere adding particular underscores; it overrides 'showstringspaces'
  showstringspaces=false,          % underline spaces within strings only
  showtabs=false,                  % show tabs within strings adding particular underscores
  stepnumber=2,                    % the step between two line-numbers. If it's 1, each line will be numbered
  stringstyle=\color{mymauve},     % string literal style
  tabsize=2,	                   % sets default tabsize to 2 spaces
 % title=\lstname                   % show the filename of files included with \lstinputlisting; also try caption instead of title
}


\setlength\headheight{20pt}
\addtolength\topmargin{-10pt}
\addtolength\footskip{20pt}

\newcommand{\N}{\mathbb{N}}
\newcommand{\ch}{\mathcal{CH}}
\everymath{\displaystyle}
\newcommand{\define}[2]{\noindent{\bf #1}}
\newcommand{\scg}{System Validation}

\newcommand{\action}[2]{{\tt #1(#2)}}
\newcommand{\todo}[1]{{\color{red}#1}}

\fancypagestyle{plain}{%
	\fancyhf{}
	\fancyhead[LO,RE]{\sffamily\bfseries\large Technische Universiteit Eindhoven}
	\fancyhead[RO,LE]{\sffamily\bfseries\large 2IMF30 \scg}
	\fancyfoot[LO,RE]{\sffamily\bfseries\large Department of Mathematics and Computer Science}
	\fancyfoot[RO,LE]{\sffamily\bfseries\thepage}
	\renewcommand{\headrulewidth}{0pt}
	\renewcommand{\footrulewidth}{0pt}
}

\pagestyle{fancy}
\fancyhf{}
\fancyhead[RO,LE]{\sffamily\bfseries\large Technische Universiteit Eindhoven}
\fancyhead[LO,RE]{\sffamily\bfseries\large 2IMF30 - System Validation}
\fancyfoot[LO,RE]{\sffamily\bfseries\large Department of Mathematics and Computer Science}
\fancyfoot[RO,LE]{\sffamily\bfseries\thepage}
\renewcommand{\headrulewidth}{1pt}
\renewcommand{\footrulewidth}{0pt}

%-------------------------------- Title ----------------------------------

\title{\sffamily\bfseries 2IMF30 \scg\ - Project 1}
\author{Tom van Diggelen \qquad Student number: 0745801 \\{\tt t.w.t.v.diggelen@student.tue.nl}\\ \\ Huib Donkers \qquad Student number: 0769015 \\{\tt h.t.donkers@student.tue.nl} \\ \\ Jeroen Noten \qquad Student number: 0784113 \\{\tt j.f.h.noten@student.tue.nl}\\ \\ Mart Pluijmaekers \qquad Student number: 0753117 \\{\tt m.h.l.pluijmaekers@student.tue.nl} \\ \\ Hein van Beers \qquad Student number: 0765658 \\{\tt h.a.v.beers@student.tue.nl}}

\date{\today}

%--------------------------------- Text ----------------------------------

\begin{document}
\maketitle
\tableofcontents

\newpage
\section{Introduction}
% !TEX root = ../report.tex
The goal of the assignment is to apply the techniques and tools learnt to the design of a software controller of a small distributed and/or embedded system. The purpose is to design this system such that it can be proven to comply with all the requirements which have been formulated in advance.\\

At first, we will give a short textual description of the system that we chose to design and implement. After that we will give a list of requirements for the whole system, that we derived from the stated description. These requirements are described in natural language. When we have settled all the requirements, we continue with identifying the interactions that are relevant to the system. We describe clearly but compactly the meaning of each interaction in words, together with its desired parameters. When the interactions have been described, we can start with translating all the global requirements in terms of these interactions. When the requirements have been described in this manner, we can start using the modal $\mu$-calculus to convert the requirements into modal formulas with which we can verify the system. After this, we will describe the behaviour of all controllers in the architecture using mCRL2\cite{url:mcrl}. This gives a short overview of all the parallel processes that communicate in our system. Finally, we will argue which requirements we have verified and which requirements we were unable to verify.

\section{Assignment description}
The assignment is about the logistic control of silicon wafers in an EUV (Extreme Ultra Violet) ASML waferstepper. If a rack of wafers arrives at an EUV waferstepper, they must enter the machine one by one trough a sluice. There are two sluices, such that if one of the two sluices does not function the other can be used to let wafers enter and leave the machine.\\

There is one robot which picks wafers from the input rack and puts them into one of the two sluices. There is another robot inside the machine which takes wafers from the sluices and puts them into a waiting rack. A third robot moves the wafers from the waiting rack to the image projection system and moves them back to another waiting rack after an image has been projected. The wafers are subsequently moved through a sluice to the outside of the machine and to an output rack.\\

It is undesirable that the machine is out of production. So, under no circumstance it is allowed that a robot damages a door of a sluice, or the vacuum in the machine is destroyed because the two doors of a sluice are open at the same time. As wafers are extremely delicate, it is also not allowed to put two wafers on top of each other.\\

Unfortunately, the doors of the sluices are less reliable than the other parts of the machine, this means that they sometimes get stuck (due to the forces exerted on these doors by atmospheric pressure). If this happens, all wafers must enter and leave the system through a single sluice, and the controller must take care that this happens automatically, without unnecessary delay.

\begin{figure}[h]
    \centering
	\includegraphics[width=0.6\textwidth]{waferstepper.png}
	\caption{\\P: Projector\\R3: Robot that places wafers under the projector\\I: Waiting rack for unprocessed wafers\\O: Waiting rack for processed wafers\\R2: Robot that moves wafers from sluices to the waiting racks and vice versa\\S1 and S2: Sluices, each sluice has an inside door I, an outside door O and a pump P\\R1: Robot that moves wafers from the input rack to the sluices and from the sluices to the output rack\\RI: Wafer input rack\\RO: Wafer output rack}
\end{figure}

\section{Basic requirements}
% !TEX root = ../report.tex
\subsection{Definitions}
\begin{enumerate}
  \item A wafer is considered projected when the projector has finished projecting the wafer.
  \item A wafer is considered processed when the wafer is projected and the wafer has been placed in \textit{RO}.
\end{enumerate}

\subsection{Assumptions}
\begin{enumerate}
  \item The doors are the only part of the machines that can malfunction. They can get stuck at any point in time. Since opening and closing a door is assumed to be an atomic action taking no time, doors only get stuck in either an open position or a closed position, never a half open position.
  \item The main room is assumed to always be a vacuum.
  \item The external input and output racks are assumed to be infinite in size.
  \item Sluice pumps change the air pressure to either zero or one atmosphere.
\end{enumerate}

\subsection{Requirements}
\begin{enumerate}
  \item At least one door of a sluice is closed at any point in time.
  \item No robot may interact with a sluice whenever its access door is closed.
  \item Any wafer in the system will eventually be processed.
  \item Internal racks, sluices, and the projector place each contain at most one wafer.
  \item When the projector is at work, no interaction with the wafer is permissible.
  \item A sluice door cannot open until the pressure on both sides is equal.
  \item Sluice pumps will not operate until both of its doors are closed.
  \item No robot can place a projected wafer in \textit{RI}.
  \item No robot can place an unprojected wafer in \textit{RO}.
  \item No robot can take a wafer from \textit{RO}.
\end{enumerate}


\section{Actions}
% !TEX root = ../report.tex

\begin{tabular}{|l|p{6cm}|p{5cm}|}
\hline  
  \textbf{Action} & \textbf{Variables} & \textbf{Meaning} \\
  \hline
  \action{move}{$a,b$} & $a,b$: A place in the system & Get a wafer from place $a$ and put it in place $b$\\
  \hline
  \action{beginProject}{} & None & Start projection of a wafer\\
  \hline
  \action{endProject}{} & None & End projection of a wafer\\
  \hline
  \action{openDoor}{$d, s$} & $d$: inside or outside door, $s$: The corresponding sluice & Open door $d$ of sluice $s$\\
  \hline
  \action{closeDoor}{$d, s$} & $d$: inside or outside door, $s$: The corresponding sluice & Close  door $d$ of sluice $s$\\
  \hline
  \action{Vacuum}{$s$} & $s$: The corresponding sluice & Start making sluice $s$ into a vacuum\\
  \hline
  \action{deVacuum}{$s$} & $s$: The corresponding sluice & Start making sluice $s$ into normal air pressure\\
  \hline
  \action{stopPumping}{$s$} & $s$: The corresponding sluice & Stop changing air pressure of sluice $s$\\
  \hline
  \action{readAirPressure}{$s, p$} & $s$: The corresponding sluice, $p \in \{$vacuum, normal$\}$: meassured air pressure & A sensor signals the airpressure in sluice $s$ is equal to inside the stepper ($p=$vacuum), or equal to outside the stepper ($p=$normal)\\
  \hline
  \action{doorStuck}{$d, s$} & $d$: inside or outside door, $s$: The corresponding sluice & A sensor signals the door is stuck and can no long move.\\
  \hline
%  \action{detectWafer}{$a, s$} & $a$: A place in the system; $s \in \{ \text{project}, \text{unprojected}, \text{empty} \}$ & Detect a projected, unprojected, or no wafer at place $a$ \\
%  \hline
  \action{detectInputWafer}{} & None & A sensor signals that an input wafer is available $RI$ \\
  \hline
\end{tabular}

\section{Requirements in terms of actions}
% !TEX root = ../report.tex

\begin{description}
 \item[At least one door of a sluice is closed at any point in time] \hfill \\
 For any sluice $s$:
 \begin{itemize}
  \item Between any \action{openInside}{$s$} and its most recent preceding \action{openOutside}{$s$}, there must be a \action{closeOutside}{$s$}
  \item Between any \action{openOutside}{$s$} and its most recent preceding \action{openInside}{$s$}, there may not be a \action{closeInside}{$s$}
 \end{itemize}

 \item[No robot may interact with a sluice whenever its access door is closed] \hfill \\
For any sluice $s$:

\begin{itemize}
	\item Between any \action{move}{$RI, s$} and its most recent preceding \action{closeOutside}{$s$}, there must be a \action{openOutside}{$s$}.
	\item between any \action{move}{$s, a$} with $a \in \{ I, O \}$ and its most recent preceding \action{closeInside}{$s$}, there must be a \action{openInside}{$s$}.
	\item between any \action{move}{$a, s$} with $a \in \{ I, O \}$ and its most recent preceding \action{closeInside}{$s$}, there must be a \action{openInside}{$s$}.
	\item between any \action{move}{$s, a$} with $a \in \{ RI, RO \}$ and its most recent preceding \action{closeOutside}{$s$}, there must be a \action{openOutside}{$s$}.
\end{itemize}
 
 \item[Any wafer in the system will eventually be processed] \hfill \\
 
 \item[Internal racks, sluices and the projector each contain at most one wafer] \hfill \\
 For any internal place $p_i \in \{S1, S2, I, O, P\}$ and any three places $p_1, p_2, p_3 \in \{S1, S2, I, O, P, RI, RO\}$: between any \action{move}{$p_1, p_i$} and its most recent preceding \action{move}{$p_2, p_i$}, there must be a \action{move}{$p_i, p_3$}.
 
 \item[When the projector is at work, no interaction with the wafer is permissible] \hfill \\
 For any place $p$: between any \action{move}{$P, p$} or \action{move}{$p, P$} and its most recent preceding \action{beginProject}{}, there must be a \action{endProject}{}.
 
 \item[A sluice door cannot open until the pressure on both sides is equal] \hfill \\
 For any sluice $s$:
 \begin{enumerate}
  \item Between any \action{openInside}{$s$} and its most recent preceding \action{deVacuum}{$s$}, there must be a \action{Vacuum}{$s$}
  \item Between any \action{openOutside}{$s$} and its most recent preceding \action{Vacuum}{$s$}, there must be a \action{deVacuum}{$s$}
 \end{enumerate}

 \item[Sluice pumps cannot operate until both of its doors are closed] \hfill \\
 For any sluice $s$:
 \begin{enumerate}
  \item Between any \action{Vacuum}{$s$} or \action{deVacuum}{$s$} and its most recent preceding \action{openInside}{$s$}, there must be a \action{closeInside}{$s$}
  \item Between any \action{Vacuum}{$s$} or \action{deVacuum}{$s$} and its most recent preceding \action{openOutside}{$s$}, there must be a \action{closeOutside}{$s$}
 \end{enumerate}

\item[No robot can place a projected wafer in $RI$] \hfill \\
For any place $a$ in the system: before any \action{move}{$a, RI$}, a \action{detectWafer}{$a, \text{unprojected}$} must have been executed and in between, no \action{move}{$a, b$} has been executed for any $b$.

\item[No robot can place an unprojected wafer in $RO$] \hfill \\
For any place $a$ in the system: before any \action{move}{$a, RO$}, a \action{detectWafer}{$a, \text{projected}$} must have been executed and in between, no \action{move}{$a, b$} has been executed for any $b$.

\item[No robot can take a wafer from $RO$] \hfill \\
\action{move}{$RO, a$} cannot take place for any $a$.
 
\end{description}


\section{Requirements in modal $\mu$-calculus}
% !TEX root = ../report.tex

\begin{description}
 \item[1. At least one door of a sluice is closed at any point in time]
	\begin{align*}
		&[true^*.openOutside(s).\overline{closeOutside(s)^*}.openInside(s)]false \\
		&[true^*.openInside(s).\overline{closeInside(s)^*}.openOutside(s)]false
	\end{align*}

 \item[2. No robot may interact with a sluice whenever its access door is closed]
	\begin{align*}
		&[true*.closeOutside(s).\overline{openOutside(s)^*}.move(RI,s)]false \\
		&[true*.closeInside(s).\overline{openInside(s)^*}.move(s,I)]false \\
		&[true*.closeInside(s).\overline{openInside(s)^*}.move(s,O)]false \\
		&[true*.closeInside(s).\overline{openInside(s)^*}.move(I,s)]false \\
		&[true*.closeInside(s).\overline{openInside(s)^*}.move(O,s)]false \\
		&[true*.closeOutside(s).\overline{openOutside(s)^*}.move(s,RI)]false \\
		&[true*.closeOutside(s).\overline{openOutside(s)^*}.move(s,RO)]false
	\end{align*}
 
 \item[3. Any wafer in the system will eventually be processed]

	\begin{align*}
		&[true^*.detectInputWafer(s)]true\\
		&[true^*.detectInputWafer(s).\overline{detectInputWafer(t)^*}]false\\
		&[true^*.detectInputWafer(s).\overline{move(RI, a)^*}]false\\
		&[true^*.move(RI, a).\overline{move(b,RO)^*}]false
	\end{align*}

 
 \item[4. Internal racks, sluices and the projector each contain at most one wafer]
    
    For all $p$, there exist $p_1, p_2, p_3$:
	\begin{align*}
		&[true^*.move(p_1, p).\overline{move(p,p_3)^*}.move(p_2,p)]false
	\end{align*}

 
 \item[5. When the projector is at work, no interaction with the wafer is permissible]
 	\begin{align*}
 		&[true^*.beginProject().\overline{endProject()^*}.move(P,O)]false
	\end{align*}
	
 \item[6. A sluice door cannot open until the pressure on both sides is equal]
	\begin{align*}
		&[true^*.vacuum(s).stopPumping(s).\overline{readAirPressure(s,0)^*}.openInside(s)]false \\
		&[true^*.deVacuum(s).stopPumping(s).\overline{readAirPressure(s,1)^*}.openOutside(s)]false \\
	\end{align*}
	
 \item[7. Sluice pumps cannot operate until both of its doors are closed]
 
  For $s \in \{sluice1, sluice2\}:$
 \begin{align*}
		&[true^*.openInside(s).\overline{closeInside(s)^*}.vacuum(s)]false \\
		&[true^*.openInside(s).\overline{closeInside(s)^*}.deVacuum(s)]false \\
		&[true^*.openOutside(s).\overline{closeOutside(s)^*}.vacuum(s)]false \\
		&[true^*.openOutside(s).\overline{closeOutside(s)^*}.deVacuum(s)]false \\
	\end{align*}

 \item[8. No robot can place a projected wafer in $RI$]\mbox{}\\
$
[true^* \cdot detectInputWafer]\\
\cdot \nu X(position:Place=RI, projecting:Bool=false, projected:Bool=false)\\
.\forall a,b:Place.[move(a,b)]X(if(a \approx position, b, position), projecting \wedge a \not \approx position, projected)\\
\wedge [\overline{\exists a,b:Place.move(a,b) + beginProject + endProject)}]X(position, projecting, projected)\\
\wedge [beginProject]X(position, position \approx P, projected)\\
\wedge [endProject]X(position, false, projected \vee (projecting \wedge position \approx P))\\
\wedge (position \approx RI \implies \neg projected)
$
 \item[9. No robot can place an unprojected wafer in $RO$] \mbox{}\\
$
[true^* \cdot detectInputWafer]\\
\cdot \nu X(position:Place=RI, projecting:Bool=false, projected:Bool=false)\\
.\forall a,b:Place.[move(a,b)]X(if(a \approx position, b, position), projecting \wedge a \not \approx position, projected)\\
\wedge [\overline{\exists a,b:Place.move(a,b) + beginProject + endProject)}]X(position, projecting, projected)\\
\wedge [beginProject]X(position, position \approx P, projected)\\
\wedge [endProject]X(position, false, projected \vee (projecting \wedge position \approx P))\\
\wedge (position \approx RO \implies projected)
$

 \item[10. No robot can take a wafer from $RO$]

\[
	\forall p:Place . [true^* \cdot move(RO, p)]false
\]

\end{description}


\section{Automaton defined in mCRL2}
% !TEX root = ../report.tex

\begin{enumerate}
 \item Sluice: door and pump (twice)
 \item Projector and $R3$
 \item $R2$
 \item $R1$
 \item $Position tracker$
\end{enumerate}


\section{Results}
We have checked the requirements on the model with the mCRL2 tool set. Because checking of some requirements takes very long, we have not obtained all results yet. The results are:
\begin{tabular}{|l|l|}
\hline
Requirement 1 & true \\
\hline
Requirement 2 & true \\
\hline
Requirement 3 & not yet known \\
\hline
Requirement 4 & not yet known \\
\hline
Requirement 5 & true \\
\hline
Requirement 6 & not yet known \\
\hline
Requirement 7 & true \\
\hline
Requirement 8 & not yet known \\
\hline
Requirement 9 & not yet known \\
\hline
Requirement 10 & true \\
\hline
\end{tabular}

\section{Conclusion}
% !TEX root = ../report.tex

Conclusion here

\begin{thebibliography}{9}

\end{thebibliography}

\newpage
\begin{appendices}
\section{mCRL2 model}
\lstinputlisting{../model/model.mcrl2}

\newpage
\section{mCRL2 requirements}

\foreach \n in {1,...,10}{
	\subsection{Requirement: \n}
	\lstinputlisting{../model/Requirements/Requirement\n.mcf}
}


\end{appendices}


\end{document}
